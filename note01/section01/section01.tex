\documentclass[../note01.tex]{subfiles}
\setcounter{section}{0}

\begin{document}
\section{実スカラー場}
自由な実スカラー場を考える.
\begin{align}\label{Lagrangian}
    S=\int\dd{t}L, \qquad L = \int\dd[3]{x} \mathcal{L}, \qquad
    \mathcal{L} = \frac{1}{2}\dot{\phi}^2 -\frac{1}{2}(\nabla\phi)^2 -\frac{1}{2}m^2\phi^2.
\end{align}
ここで, 計量を
\begin{align}
    g_{\mu\nu} = \mathrm{diag} (-1,1,1,1)
\end{align}
とすると, Lagrangian密度は
\begin{align}
    \mathcal{L} = -\frac{1}{2}\partial^\mu\phi\partial_\mu\phi - \frac{1}{2}m^2\phi^2
\end{align}
と書ける. この章の大雑把な流れは次のとおりである.
\begin{enumerate}
    \item \textbf{古典的に運動方程式を解く.}
    \item \textbf{正準量子化する.}
    \item \textbf{エネルギー固有状態を構成する.}
\end{enumerate}
他の章でも主にこの流れで進み, 必要に応じて数学的な準備をする.
\subsection{古典論}
この節では, 正準形式で古典的に運動方程式を解いて, 解がモード展開の形で書けることを見る.

まず, $ \phi(x) $ の共役運動量を $ \pi(x) $ とすると,
\begin{align}\label{conjugate_momentum}
    \pi(\vb{x}) = \fdv{L[\phi,\dot{\phi}]}{\dot{\phi}(\vb{x})} = \dot{\phi}(\vb{x})
\end{align}
である. 分かりやすさのために時間の引数を敢えて書かなかった.
この理論は3次元空間の各点に1つずつ独立な自由度 $ \phi(\vb{x}) $ を持っていて, Lagrangianはそれらをある時刻において足し合わせたものである.
有限自由度の力学の場合の定義
\begin{align*}
    p_i = \pdv{L(q,\dot{q})}{\dot{q}^i}, \qquad q=(q^1,\dots,q^n)
\end{align*}
を思い出すと, 式 \eqref{conjugate_momentum} がその自然な一般化であることがすぐ分かるだろう.
\subsubsection{汎関数微分の定義}
ここで一応, 汎関数微分の定義を与えておく. 読み飛ばしても構わない.
一般に, 関数 $ f(x) $ の汎関数 $ I[f] $ を考えて, $ I[f+g] $ を $ f $ まわりでTaylor展開しようとすると
\begin{align*}
    I[f+g] = I[f] + \int\dd{x}\fdv{I[f]}{f(x)}g(x) + \frac{1}{2}\int\dd{x}\dd{y}\frac{\delta^2 I[f]}{\delta f(x)\delta f(y)}g(x)g(y) + \cdots
\end{align*}
となるだろう. つまり, (1階の)汎関数微分を得るには1次の項だけ見ればよく, それは
\begin{align*}
    \int\dd{x}\fdv{I[f]}{f(x)}g(x) = \left.\dv{\epsilon}I[f+\epsilon g]\right|_{\epsilon=0}
\end{align*}
で得られる.
\begin{katei}
    $ \delta f(x) $ を任意関数とする. 汎関数 $ I[f] $ に対して
    \begin{align}
        \delta I[f] := \left.\dv{\epsilon}I[f+\epsilon\delta f]\right|_{\epsilon=0}
    \end{align}
    を(1次の)\textbf{変分}といい, これを用いて(1階の)\textbf{汎関数微分} $ \delta I/\delta f(x) $ を
    \begin{align}
        \delta I[f] =: \int\dd{x} \fdv{I}{f(x)} \delta f(x)
    \end{align}
    で定義する. 汎関数微分は, 関数の引数の逆数の次元を持つことに注意する.
\end{katei}
\noindent 例えば, Lagrangian\eqref{Lagrangian}の $ \dot{\phi}(\vb{x}) $ に関する変分は
\begin{align*}
    \delta L[\dot{\phi}] = \left.\dv{t} \int\dd[3]{x} \frac{1}{2}\qty(\dot{\phi}+t\delta\dot{\phi})^2\right|_{t=0}
    = \int\dd[3]{x} \dot{\phi}\delta\dot{\phi}
\end{align*}
なので, 式 \eqref{conjugate_momentum} の2番目の等号は正しい. また一般に, 関数 $ f(x) $ 自体を自身の汎関数だと思ったとき,
\begin{align*}
    \delta f(y) &= \delta \int\dd{x} \delta(x-y)f(x) \\
    &= \left.\dv{t} \int\dd{x} \delta(x-y) \qty[f(x)+t\delta f(x)]\right|_{t=0} \\
    &= \int\dd{x} \delta(x-y)\delta f(x)
\end{align*}
より,
\begin{align*}
    \fdv{f(y)}{f(x)} = \delta(x-y)
\end{align*}
が成り立つ. 他には, 微分を含む汎関数についても汎関数微分を考えられる. 例えば,
\begin{align*}
    I[f] = \int\dd{x} \frac{1}{2}f'(x)^2
\end{align*}
なら
\begin{align*}
    \delta I[f] = \left.\dv{t} \int\dd{x} \frac{1}{2} (f'+t\delta f')^2 \right|_{t=0}
    = \int\dd{x} f'\delta f'
\end{align*}
なので, 表面項が無視できるなら部分積分して
\begin{align*}
    \fdv{I}{f(x)} = -f''(x)
\end{align*}
となる.

\subsubsection{運動方程式とその解}
Hamiltonianは
\begin{align}
    H[\phi,\pi] &= \int\dd[3]{x} \pi(\vb{x})\dot{\phi}(\vb{x}) -L \notag \\
    &= \int\dd[3]{x} \qty[\frac{1}{2}\pi^2 + \frac{1}{2}(\nabla\phi)^2 + \frac{1}{2}m^2\phi^2]
\end{align}
となる. Poisson括弧は, 同時刻の関数 $ \phi(\vb{x}),\pi(\vb{x}) $ の汎関数 $ f,g $ に対して
\begin{align}
    \{f,g\}_\mathrm{P} = \int\dd[3]{z} \qty[\fdv{f}{\phi(\vb{z})}\fdv{g}{\pi(\vb{z})} - (\phi\leftrightarrow\pi)]
\end{align}
で定義されるので,
\begin{align}\label{Poisson_bracket}
    \{\phi(\vb{x}),\phi(\vb{y})\}_\mathrm{P} = \{\pi(\vb{x}),\pi(\vb{y})\}_\mathrm{P} = 0, \qquad
    \{\phi(\vb{x}),\pi(\vb{y})\}_\mathrm{P} = \delta^3(\vb{x}-\vb{y})
\end{align}
が成り立ち,
\begin{align*}
    \{\phi(\vb{x}),H\}_\mathrm{P} &= \int\dd[3]{z} \delta^3(\vb{z}-\vb{x})\pi(\vb{z}) = \pi(\vb{x}), \\
    \{\pi(\vb{x}),H\}_\mathrm{P} &= -\int\dd[3]{z} \delta^3(\vb{z}-\vb{x})(-\nabla^2\phi + m^2\phi)(\vb{z}) = (\nabla^2-m^2)\phi(\vb{x})
\end{align*}
が成り立つ. 2つ目の式では, \textbf{場} $ \phi(\vb{x}) $ \textbf{が遠方} $ \vb{x}\to\infty $ \textbf{で0となることを仮定}して部分積分した.
よって, 運動方程式は
\begin{align}
    \dot{\phi}(x)=\pi(x), \qquad \dot{\pi}(x)=(\nabla^2-m^2)\phi(x),
\end{align}
つまり, \textbf{Klein-Gordon方程式}
\begin{align}\label{KG-eq}
    (-\partial^2+m^2)\phi(x)=0
\end{align}
となる. この一般解はすぐに求まる. 特に $ \phi^*(x)=\phi(x) $ に注意する.
\begin{katei}
    運動方程式\eqref{KG-eq}の解は, 時間によらない任意の複素数値関数$ a(\vb{k}) $ を用いて
    \begin{align}
        \phi(x) = \int\widetilde{\dd{k}} \qty[a(\vb{k})\E^{\I kx} + a^*(\vb{k})\E^{-\I kx}], \qquad
        \widetilde{\dd{k}} := \frac{\dd[3]{k}}{(2\pi)^3 2\omega}, \qquad
        k^0 = \omega := \sqrt{\vb{k}^2+m^2}
    \end{align}
    と書ける.
\end{katei}
最後に, 各モード $ a(\vb{k}) $ が $ \phi(x) $ を使ってどう書けるかを見る. 先に微分記号を定義する.
\begin{align}
    f\overset{\leftrightarrow}{\partial} g := f\partial g - (\partial f)g.
\end{align}
$ k^0=\omega $ として, 次を計算してみる.
\begin{align*}
    &\quad\;\I\int\dd[3]{x} \E^{-\I kx}\overset{\leftrightarrow}{\partial}_0\phi(x) \\
    &= \I\int\dd[3]{x} \E^{-\I kx}\overset{\leftrightarrow}{\partial}_0 \int\widetilde{\dd{k'}} \qty[a(\vb{k}')\E^{\I k'x} + a^*(\vb{k}')\E^{-\I k'x}] \\
    &= \I\E^{\I\omega t}\overset{\leftrightarrow}{\partial}_0 \int\widetilde{\dd{k}'} \qty[a(\vb{k}')\E^{-\I\omega't} + a^*(-\vb{k}')\E^{\I\omega't}] \int\dd[3]{x}\E^{-\I(\vb{k}-\vb{k}')\vb{x}} \\
    &= \I\E^{\I\omega t}\overset{\leftrightarrow}{\partial}_0 \frac{1}{2\omega} \qty[a(\vb{k})\E^{-\I\omega t} + a^*(-\vb{k})\E^{\I\omega t}] \\
    &= \frac{\I\E^{\I\omega t}}{2\omega} \qty[(-\I\omega-\I\omega)a(\vb{k})\E^{-\I\omega t} + (\I\omega-\I\omega)a^*(\vb{-k})\E^{\I\omega t}] \\
    &= a(\vb{k}).
\end{align*}
\begin{kekka}
    $ \phi(x) $ のFourierモードは, $ \phi(x) $ から
    \begin{align}\label{mode}
        a(\vb{k}) = \I\int\dd[3]{x} \E^{-\I kx}\overset{\leftrightarrow}{\partial}_0\phi(x), \qquad k^0=\omega
    \end{align}
    で得られる.
\end{kekka}

\subsection{正準量子化}
次のようにして理論を正準量子化する.
\begin{katei}
    場 $ \phi(x) $ を演算子として再解釈し, 同時刻交換関係をPoisson括弧\eqref{Poisson_bracket}の $\I$ 倍で定義する.
    \begin{align}
        [\phi(t,\vb{x}),\phi(t,\vb{x}')] = [\pi(t,\vb{x}),\pi(t,\vb{x}')] = 0, \qquad
        [\phi(t,\vb{x}),\pi(t,\vb{x}')]=\I\delta^3(\vb{x}-\vb{x}').
    \end{align}
    特に今の場合, $ \phi(x) $ はHermite演算子であると仮定する.
    \begin{align}
        \phi^\dagger(x) = \phi(x).
    \end{align}
\end{katei}
\noindent このとき, Fourierモードの表式\eqref{mode}から, $ a^*(\vb{k}) $ は $ a^\dagger(\vb{k}) $ で置き換えられることが分かる.
次に, $ \phi(x),\pi(x) $ の交換関係から $ a(\vb{k}),a^\dagger(\vb{k}) $ の交換関係を計算する.
\begin{align*}
    [a(\vb{k}),a^\dagger(\vb{k}')] &= \qty[\I\int\dd[3]{x} \E^{\I\omega t}\E^{-\I\vb{k}\vb{x}} \Bigl(\pi(t,\vb{x})-\I\omega\phi(t,\vb{x})\Bigr), -\I\int\dd[3]{x'} \E^{-\I\omega't}\E^{\I\vb{k}'\vb{x}'} \Bigl(\pi(t,\vb{x}')+\I\omega'\phi(t,\vb{x}')\Bigr)] \\
    &= \E^{\I(\omega-\omega')t} \int\dd[3]{x}\E^{-\I\vb{k}\vb{x}} \int\dd[3]{x'}\E^{\I\vb{k}'\vb{x}'} \Bigl(\I\omega'[\pi(t,\vb{x}),\phi(t,\vb{x}')] - \I\omega[\phi(t,\vb{x}),\pi(t,\vb{x}')]\Bigr) \\
    &= \E^{\I(\omega-\omega')t} \int\dd[3]{x}\E^{-\I\vb{k}\vb{x}} \int\dd[3]{x'}\E^{\I\vb{k}'\vb{x}'} (\omega'+\omega)\delta^3(\vb{x}-\vb{x}') \\
    &= \E^{\I(\omega-\omega')t} \int\dd[3]{x}\E^{-\I(\vb{k}-\vb{k}')\vb{x}} (\omega'+\omega) \\
    &= (2\pi)^3 2\omega \delta^3(\vb{k}-\vb{k}'), \\
    [a(\vb{k}),a(\vb{k}')] &= \qty[\I\int\dd[3]{x} \E^{\I\omega t}\E^{-\I\vb{k}\vb{x}} \Bigl(\pi(t,\vb{x})-\I\omega\phi(t,\vb{x})\Bigr), \I\int\dd[3]{x'} \E^{\I\omega't}\E^{-\I\vb{k}'\vb{x}'} \Bigl(\pi(t,\vb{x}')-\I\omega'\phi(t,\vb{x}')\Bigr)] \\
    &= -\E^{\I(\omega+\omega')t} \int\dd[3]{x}\E^{-\I\vb{k}\vb{x}} \int\dd[3]{x'}\E^{-\I\vb{k}'\vb{x}'} \Bigl(-\I\omega'[\pi(t,\vb{x}),\phi(t,\vb{x}')] - \I\omega[\phi(t,\vb{x}),\pi(t,\vb{x}')]\Bigr) \\
    &= -\E^{\I(\omega+\omega')t} \int\dd[3]{x}\E^{-\I\vb{k}\vb{x}} \int\dd[3]{x'}\E^{-\I\vb{k}'\vb{x}'} (-\omega'+\omega)\delta^3(\vb{x}-\vb{x}') \\
    &= \E^{\I(\omega+\omega')t} \int\dd[3]{x}\E^{-\I(\vb{k}+\vb{k}')\vb{x}} (\omega'-\omega) \\
    &= 0.
\end{align*}
\begin{kekka}
    $ a(\vb{k}),a^\dagger(\vb{k}) $ は交換関係
    \begin{align}
        [a(\vb{k}),a^\dagger(\vb{k}')] = (2\pi)^3 2\omega\delta^3(\vb{k}-\vb{k}'),\qquad [a(\vb{k}),a(\vb{k}')]=[a^\dagger(\vb{k}),a^\dagger(\vb{k}')]=0
    \end{align}
    を満たす.
\end{kekka}

\subsection{エネルギースペクトル}
この節では, Noetherの定理を使ってエネルギー運動量テンソルを古典的に求め, 次に量子論で $ a^\dagger(\vb{k}),a(\vb{k}) $ が生成消滅演算子の役割を果たすことを見る.
\subsubsection{対称性とは何か}
先に, 古典論における対称性とは何かということを時空並進を例に取って少し考えてみる.
\begin{align}
    x' = x + a.
\end{align}
$ \phi $ はスカラー場なので, 同一時空点では同じ値を持つように変換する.
\begin{align}
    \phi'(x')=\phi(x).
\end{align}
ここで, 同一時空点が $ x $ に見える人と $ x' $ に見える人を用意して, 前者をAさん, 後者をBさんと呼ぶことにする.
この場合, BさんはAさんから $ a $ だけ下がった位置にいることになる.
もし, この系が考えている変換の対称性を持つことを言いたければ, Aさんから見た物理とBさんから見た物理が同じ, つまり
\begin{align}
    S[\phi'] = S[\phi]
\end{align}
であることが言えればいいだろう. ただし, Aさんからの見え方とBさんからの見え方が同じならば, 作用の積分範囲も同じでなくてはならないことに注意する. この場合, 実際
\begin{align*}
    S[\phi'] &= \int\dd[4]{x} \qty[-\frac{1}{2}(\partial\phi'(x))^2 - \frac{1}{2}m^2\phi'(x)^2] \\
    &= \int\dd[4]{x'} \qty[-\frac{1}{2}(\partial'\phi'(x'))^2 - \frac{1}{2}m^2\phi'(x')^2] \\
    &= \int\dd[4]{x} \qty[-\frac{1}{2}(\partial\phi(x))^2 - \frac{1}{2}m^2\phi(x)^2] \\
    &= S[\phi]
\end{align*}
なので, この系が並進対称性を持つことが分かった.

\subsubsection{エネルギー運動量テンソルの導出}
並進による場の変化を変換のパラメータ $ a^\mu $ で展開すると
\begin{align*}
    \phi'(x)-\phi(x) = \phi(x-a)-\phi(x) = -a^\mu\partial_\mu\phi(x)+\order{a^2}
\end{align*}
となるので, これを元に
\begin{align}
    \delta\phi(x)=-\epsilon^\mu(x)\partial_\mu\phi(x).
\end{align}
という関数を考える. ここで, $ \epsilon^\mu(x) $ を $ t\to\pm\infty $ で0になる任意関数とすると, $ \delta\phi(x) $ も同様に$ t\to\pm\infty $ で0になる任意関数となる.
これを使って作用の変分を取ると,
\begin{align*}
    \delta S[\phi] &= \int\dd[4]{x} \qty[-\partial^\mu\phi\partial_\mu\delta\phi - m^2\phi\delta\phi] \\
    &= \int\dd[4]{x} \qty[-\partial^\mu\phi\partial_\mu(-\epsilon_\nu\partial^\nu\phi) - m^2\phi(-\epsilon_\nu\partial^\nu\phi)] \\
    &= \int\dd[4]{x} \qty[\partial_\mu\epsilon_\nu\partial^\mu\phi\partial^\nu\phi - \epsilon_\nu\qty(-\partial^\mu\phi\partial_\mu\partial^\nu\phi - m^2\phi\partial^\nu\phi)] \\
    &= \int\dd[4]{x} \qty[-\epsilon_\nu\partial_\mu(\partial^\mu\phi\partial^\nu\phi) -\epsilon_\nu\partial^\nu\qty(-\frac{1}{2}\partial^\mu\phi\partial_\mu\phi-\frac{1}{2}m^2\phi^2)] \\
    &= -\int\dd[4]{x} \epsilon_\nu\partial_\mu\qty[\partial^\mu\phi\partial^\nu\phi + \eta^{\mu\nu}\mathcal{L}]
\end{align*}
となる. 4つ目の等号では, $ \epsilon^\mu(x) $ のおかげで時間に関する部分積分も行えた.
ここで, 最小作用の原理から, 運動方程式を満たす場 $ \phi(x) $ は $ \delta S[\phi]=0 $ を満たすので, $ \epsilon^\mu(x) $ の $ -\infty<t<\infty $ での任意性によって次が言える.
\begin{kekka}
    運動方程式を満たす場 $ \phi(x) $ は
    \begin{align}\label{EMT}
        \partial_\mu T^{\mu\nu}=0, \qquad T^{\mu\nu} = \partial^\mu\phi\partial^\nu\phi + \eta^{\mu\nu}\mathcal{L}
    \end{align}
    を満たす. この $ T^{\mu\nu} $ を(正準)\textbf{エネルギー運動量テンソル}という.
\end{kekka}

\subsubsection{スペクトルの計算}
量子論に移る.
\begin{katei}
    量子論でもエネルギー運動量テンソルをそのまま\eqref{EMT}で定義し, 4元運動量の演算子を
    \begin{align}
        P^\mu = \int\dd[3]{x}T^{0\mu}
    \end{align}
    とする.
\end{katei}
\noindent このとき, $ P^\mu $ を $ a^\dagger(\vb{k}),a(\vb{k}) $ で展開すると,
\begin{align*}
    P^0 &= \int\dd[3]{x} T^{00} = \int\dd[3]{x} (\dot{\phi}-\mathcal{L}) \qquad (= H) \\
    &= \frac{1}{2}\int\dd[3]{x} \qty[\dot{\phi}^2 + (\nabla\phi)^2 + m^2\phi^2] \\
    &= \frac{1}{2}\int\dd[3]{x} \qty[\dot{\phi}^2 + \phi(-\nabla^2+m^2)\phi] \\
    &= \frac{1}{2}\int\dd[3]{x}\widetilde{\dd{k}}\widetilde{\dd{k'}} \biggl\{\I\omega\Bigl[a(\vb{k})\E^{-\I\omega t} - a^\dagger(-\vb{k})\E^{\I\omega t}\Bigr] \cdot \I\omega'\qty[a(-\vb{k}')\E^{-\I\omega't} - a^\dagger(\vb{k}')\E^{\I\omega't}] \\
    &\qquad \Bigl[a(\vb{k})\E^{-\I\omega t} + a^\dagger(-\vb{k})\E^{\I\omega t}\Bigr] \cdot (\vb{k}'^2+m^2) \qty[a(-\vb{k}')\E^{-\I\omega't} + a^\dagger(\vb{k}')\E^{\I\omega't}]\biggr\} \E^{\I(\vb{k}-\vb{k}')\vb{x}} \\
    &= \frac{1}{2}\int\widetilde{\dd{k}}\frac{1}{2\omega} \cdot \omega^2 \Bigl\{-\qty[a(\vb{k})\E^{-\I\omega t} - a^\dagger(-\vb{k})\E^{\I\omega t}] \qty[a(-\vb{k})\E^{-\I\omega t} - a^\dagger(\vb{k})\E^{\I\omega t}] \\
    &\hspace{32.5mm} \qty[a(\vb{k})\E^{-\I\omega t} + a^\dagger(-\vb{k})\E^{\I\omega t}] \qty[a(-\vb{k})\E^{-\I\omega t} + a^\dagger(\vb{k})\E^{\I\omega t}]\Bigr\} \\
    &= \frac{1}{2}\int\widetilde{\dd{k}}\; \omega \qty[a(\vb{k})a^\dagger(\vb{k}) + a^\dagger(\vb{k})a(\vb{k})] \\
    &= \frac{1}{2}\int\widetilde{\dd{k}}\; \omega \qty[2a^\dagger(\vb{k})a(\vb{k}) + (2\pi)^3 2\omega\delta^3(\vb{k}=0)] \\
    &= \int\widetilde{\dd{k}}\; \omega a^\dagger(\vb{k})a(\vb{k}) + \int\frac{\dd[3]{x}\dd[3]{k}}{(2\pi)^3} \frac{\omega}{2}, \\
    P^i &= \int\dd[3]{x} T^{0i} = -\int\dd[3]{x} \dot{\phi}\partial_i\phi \\
    &= -\int\dd[3]{x}\widetilde{\dd{k}}\widetilde{\dd{k'}}\; \qty(-\I\omega) \Bigl[a(\vb{k})\E^{-\I\omega t} - a^\dagger(-\vb{k})\E^{\I\omega t}\Bigr] \cdot \qty(-\I k'^i)\qty[a(-\vb{k}')\E^{-\I\omega't} + a^\dagger(\vb{k}')\E^{\I\omega't}] \E^{\I(\vb{k}-\vb{k}')\vb{x}} \\
    &= \int\widetilde{\dd{k}} \frac{1}{2\omega} \cdot \omega k^i \qty[a(\vb{k})\E^{-\I\omega t} - a^\dagger(-\vb{k})\E^{\I\omega t}] \qty[a(-\vb{k})\E^{-\I\omega t} + a^\dagger(\vb{k})\E^{\I\omega t}] \\
    &= \frac{1}{2}\int\widetilde{\dd{k}}\; k^i \qty[a(\vb{k})a^\dagger(\vb{k}) - a^\dagger(-\vb{k})a(-\vb{k})] \\
    &= \frac{1}{2}\int\widetilde{\dd{k}}\; k^i \qty[a(\vb{k})a^\dagger(\vb{k}) + a^\dagger(\vb{k})a(\vb{k})] \\
    &= \int\widetilde{\dd{k}}\; k^i a^\dagger(\vb{k})a(\vb{k})
\end{align*}
となるので, $ P^\mu $ と $ a^\dagger(\vb{k}),a(\vb{k}) $ の交換関係は, $ k^0=\omega $ として,
\begin{align*}
    \qty[P^\mu,a^\dagger(\vb{k})] &= \int\widetilde{\dd{k'}}\; k'^\mu [a^\dagger(\vb{k}')a(\vb{k}'),a^\dagger(\vb{k})] \\
    &= \int\widetilde{\dd{k'}}\; k'^\mu a^\dagger(\vb{k}')\; (2\pi)^3 2\omega' \delta^3(\vb{k}'-\vb{k}) \\
    &= k^\mu a^\dagger(\vb{k}), \\
    \qty[P^\mu,a(\vb{k})] &= -\qty[P^\mu,a^\dagger(\vb{k})]^\dagger \\
    &= -k^\mu a(\vb{k})
\end{align*}
となる.
\begin{kekka}
    4元運動量 $ P^\mu $ は, c数を除いて
    \begin{align}
        P^\mu = \int\widetilde{\dd{k}}\; k^\mu a^\dagger\vb(\vb{k})a(\vb{k}), \qquad k^0 = \omega
    \end{align}
    と書けて,
    \begin{align}
        \qty[P^\mu,a^\dagger(\vb{k})] = k^\mu a^\dagger(\vb{k}), \qquad \qty[P^\mu,a(\vb{k})] = -k^\mu a(\vb{k}), \qquad k^0=\omega
    \end{align}
    を満たすので, $ a^\dagger(\vb{k}),a(\vb{k}) $ は $ P^\mu $ の固有値を $ k^\mu $ だけそれぞれ上げ下げする.
    よって, $ a^\dagger(\vb{k}),a(\vb{k}) $ はそれぞれ\textbf{生成消滅演算子}であり, 真空 $ \ket{0} $ を
    \begin{align}
        P^\mu\ket{0} = 0 \qquad\text{かつ}\qquad \forall\vb{k}, \; a(\vb{k})\ket{0} = 0
    \end{align}
    な状態として定義すると, 4元運動量の固有状態は
    \begin{alignat*}{2}
        \text{基底状態}&\quad \ket{0} & &\quad :\quad P^\mu\ket{0} = 0 \\
        \text{1粒子状態}&\quad \ket{\vb{k}} = a^\dagger(\vb{k})\ket{0} & &\quad :\quad P^\mu\ket{\vb{k}} = k^\mu\ket{\vb{k}},\; k^0=\omega 
    \end{alignat*}
    などである.
\end{kekka}

\end{document}